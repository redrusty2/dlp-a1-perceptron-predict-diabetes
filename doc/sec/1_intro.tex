\section{Introduction}
\label{sec:intro}

In 2021, it was estimated that 537 million people worldwide had diabetes, and this number was projected to increase to 643 million by 2030 and 783 million by 2045. It was also approximated that 45\% of people with diabetes, predominantly those with type 2 diabetes, were undiagnosed, highlighting an urgent need to improve diagnosis and provide timely care(IDF and cgpt).

Predicting diabetes is a binary classification problem. A binary classification is the process of categorising data into 2 classes.  

We will use the Pima Indians Diabetes dataset which has been used in many other papers to measure the performance of different algorithms

Before machnine learning, the most common approach to binary classification problems was to use statistical methods.

We will be measuring the performance of the perceptron model. The perceptron model is a supervised machine learning algorith for learning a linear classifier. 

The previous approaches have been to use statistical methods, which require a lot of computational power. LDA and logistic regression are examples of these methods.

What are some other downsides? hidden relationships. Bad with small datasets. underlying relationships not known. Complex interactions and intercorrelations among many features.

Limitations of perceptron. Linearly separable data. No probabilistic output.

There are many better ways to do this now. The purpose of this paper is to analyse the performance of the perceptron algorithm on the Pima Indians Diabetes dataset.
